\documentclass[runningheads]{llncs}
\usepackage{graphicx,amssymb,pepa,pgfplots,pgfplotstable,amsmath,placeins,listings}
\usetikzlibrary{patterns}

% ---- Highlighting placeholder text
\usepackage{xcolor}
\usepackage{framed}
\colorlet{shadecolor}{red!20}

\begin{document}
\title{Performance modelling and simulation of skewed demand in complex systems}

\author{Stephen Shephard}

\institute{School of Computing Science, Newcastle University, Newcastle upon Tyne, NE1 7RU\\
	\email{s.shephard2@newcastle.ac.uk}}

\maketitle

%
% ---- Abstract and keywords
%
\begin{abstract}
On-line Transaction Processing (OLTP) applications must frequently deal with the issue of skewed demand for some resources.  This demand may overwhelm the whole system, affecting the owner's reputation and revenue.  This paper presents system architectures for a ticketing use case using a selection of distributed computing technologies of the Cloud.  It proposes models of these architectures and uses them to predict throughput in skewed demand scenarios.  The experimental results of the models are then tested against simple built systems.  It finds that the models correctly predict that a shared queue constrains throughput in proportion to relative demand, but overestimate the gains from a distributed database with replication.
	\keywords{Cloud, middleware, microservices, distributed databases, modelling, performance}
\end{abstract}

%
% ---- Introduction
%

\section{Introduction}\label{sec:introduction}

There are many high-profile examples of whole IT systems brought down by customer demand for part of their services.  Customers were prevented from using any part of the London 2012 Olympic ticketing website on launch day to avoid demand overloading the system \cite{RN1067}.  Demand for the finale of `True Detective' \cite{RN1066} brought down HBO Go.  Apple's iTunes Store suffered outage on the launch day of the iPhone 7 (new iPhone registration is carried out via an iTunes function) \cite{RN1068}.

This paper claims that it is possible to design and build more resilient systems through effective use of Cloud technologies where higher than normal demand for one function or type of resource would not block access to the others.  Skewed demand may be isolated so that it only affects parts of a system, or shared equally between different components. (The system may also adapt to demand by elastic scaling of resources, but this will not be considered as part of this paper).  When combining a middleware solution with a distributed database, however, where is the system bottleneck? If there are levels of demand that cannot be met on a limited budget, and that therefore some components will no longer meet the required throughput, what is the impact on the remainder of the system?

\subsection{Aims and Objectives}

The aim is to demonstrate that PEPA (Performance Evaluation Process Algebra) \cite{RN1051} may be used to construct models that are good predictors of the behaviour of real systems under skewed demand scenarios, and provide insights into the effectiveness of Cloud technologies for handling skewed demand.  The objectives are to:
\begin{itemize}
	\item show that distributed databases and middleware queues may be modelled in PEPA.
	\item show that these PEPA component models may be used to test skewed demand.
	\item show that the component models may be composed into system models with interesting behaviour under test.
	\item build actual system architectures for the PEPA models and show that the built systems behaviours match those predicted by the models.
\end{itemize}

\subsection{Methodology}

Performance models will be developed using a bottom-up approach from components to system architectures.  These will be iteratively tuned following development and measurement of built systems.

The paper is organised as follows.  Section \ref{sec:background} presents a use case based on ticket booking for a multi-sport event.  Section \ref{sec:technologies} selects a set of distributed technologies, and section \ref{sec:modelling} discusses modelling in PEPA.  Simple component models are proposed and tested in section \ref{sec:pepa-component-models} and composed into more complex system models that make end-to-end predictions.
The models will then be tested against actual systems in section \ref{sec:built-systems}, built using Java and the Java Spring framework \cite{RN1076}, Cassandra \cite{RN1050}\cite{RN1075} databases, and where appropriate Microsoft Azure Storage Queues \cite{RN1072}.  These systems will be instrumented with Coda Hale Metrics \cite{RN1079} and measured under different scenarios of skewed demand, using Apache JMeter test plans \cite{RN1074}.  Finally section \ref{sec:conclusions} evaluates the approach and makes suggestions for future work.

%
% ---- Background
%

\section{Background}\label{sec:background}

Consider a general OLTP application using a distributed architecture, as shown in Figure \ref{figure:oltpapplication}.  Users access the application with a web-based front end.  Resources are stored in one or more databases.  In between the web servers and database are worker applications that service user requests, connected to the web servers by some middleware.  There are strategies for coping with skewed demand at each layer of this architecture.

\begin{figure}
	\label{figure:oltpapplication}
	\centering
	\includegraphics[trim = 5 5 5 5, clip, width=\textwidth]{img/application}
	\caption{OLTP application distributed architecture}
\end{figure}

\paragraph{Adapting.} A system using {\itshape elastic scaling} may adapt to increased demand. Rapid elasticity is an essential characteristic of Cloud Computing by the NIST definition \cite{RN56}.  Computing resources, for example web servers or worker applications, can be elastically and often automatically scaled to meet current demand.  This gives the appearance of resources that are limited only by the system owner's budget.

\paragraph{Sharing.} High demand may be shared between resources.  HTTP load balancing improves the scalability of a web-based application by distributing the demand across multiple web servers \cite{RN73}.  Shared middleware such as a point-to-point queue, provides a competing consumer pattern to balance load from several producers, e.g. web servers, between multiple consumers e.g. worker applications.

\paragraph{Isolating.} If it is not possible to satisfy the skewed demand within a given budget, then it may be appropriate to isolate that demand from the rest of the system.  Horizontal partitioning of a distributed database can place high demand resources on different data nodes.  Microservices architecture offers a pattern for partitioning the data resources, the worker applications and the web servers, connecting them into entirely separate smaller end-to-end services.

\subsection{Use Case}

The concrete use case for constructing models and building systems is a ticketing application.  Following the Olympic example given in the Introduction, tickets will be for a multi-sport event.  Some sports are more popular than others and it will be assumed that there will be predictable skewed demand for {\itshape athletics} tickets.  Use cases where the resources with skewed demand are unknown - where they are only discovered once the application goes online, requiring some adaptive approach - are out of scope.

The application has three possible operations:
\begin{enumerate}
\item Search (for available tickets)
\item Book (allocate a ticket to a customer)
\item Return (customer releases a ticket allocation)
\end{enumerate}

Such a ticketing application may be generalised to any system for allocating and releasing other resources with variable demand.

This paper considers the problem of higher than average demand for a particular type of ticket, and to what extent the system will allow users to search for other ticket types if some component is overloaded by the skewed demand for the most popular tickets.  It does not consider issues of fair allocation of scarce resources.

%
% ---- Technologies
%

\section{Technologies}\label{sec:technologies}

%
% ---- Choice of technologies in scope
%
\subsection{Scope}
\subsubsection{In Scope.}  The selected technologies in scope of this paper are shared middleware queues, distributed databases and microservices, which are discussed in more detail below.  The database partitioning strategy and entirely separate databases for the microservices architecture offer alternative means of isolating the skewed demand.  Using a single middleware queue shares and distributes the demand, this time in contrast to the microservices middleware approach.  The models will compare these approaches and investigate the behaviour of systems where the components have conflicting approaches to handling demand.
\subsubsection{Out of Scope.}  The paper will not consider elastic scaling or HTTP load balancing.  There is already a great deal of work in evaluating right-sizing strategies (minimising underutilisation and overutilisation of compute resources) for the former, e.g. \cite{RN49}, \cite{RN62}, \cite{RN48}.  HTTP load balancing is a relatively mature technology, and work has been done on simulation to evaluate different algorithms by response time and web server utilisation \cite{RN55}.

%
% ---- Queue Middleware
%

\subsection{Queue Middleware}\label{sec:middleware}

Good choice of middleware in a system will help ensure that its components are connected, but loosely coupled.  If, for example, a web server is blocked waiting for a response from a worker application carrying out a more expensive operation, then the throughput of the web server will be limited to that of the worker application.  The use case `return' operation however does not require a direct response from the system.  As long as the customer can rely on eventual guaranteed delivery of the return request, (and that the cost of their ticket will be refunded) then they do not need to wait for a direct response to their return.

Point-to-Point Queues, e.g. Azure Storage Queues \cite{RN1072}, are a form of Message-Oriented Middleware - an asynchronous, brokered message service providing an intermediate layer between senders and receivers, decoupling their communication.  Message delivery may take minutes rather than milliseconds, but the service providers do provide configurable delivery guarantees \cite{RN65}.  With synchronous middleware such as Remote Procedure Call (RPC), the calling process is blocked until the called service completes and returns control to the caller.  Distributed systems using asynchronous middleware do not block when calling a remote service.  Control is immediately passed back to the caller, and a response may be returned eventually, with the caller polling the remote service for the response, or the remote process calling a method in the caller to send the response.

Many processes may send messages to a queue, and each message is received by one consumer - though it may be one of several consumers competing for messages from this queue.  This competing consumer pattern offers a means of balancing load from the Web Servers between the Worker Applications in the ticketing use case.

%
% ---- Microservices
%
\FloatBarrier
\subsection{Microservices}\label{sec:microservices}

Microservices architecture is an approach to structuring applications as suites of small services, defined by business capability verticals rather than technological layers \cite{RN1069} \cite{RN1070}.  Each of the use case requirements - search for, book or return tickets - might typically be microservices with their own worker applications and data nodes.  Ticket data would be denormalised across the data nodes and made eventually consistent via a backplane messaging service \cite{RN1071} as shown in Figure \ref{figure:microservices}.  This would certainly isolate the demand for search, book and return from each other - returning tickets would not be blocked by a system where bookings were overloaded.  In the ticketing use case however, there is skewed demand for athletics tickets.  In a real-world system the booking microservice might be further broken down to a lower level of granularity to deal with this, i.e. a separate microservice for booking each ticket type.

\begin{figure}
	\label{figure:microservices}
	\centering
	\includegraphics[trim = 5 5 5 5, clip, width=\textwidth]{img/microservices}
	\caption{Microservices}
\end{figure}

%
% ---- Distributed databases
%
\FloatBarrier
\subsection{Distributed databases}\label{sec:distributed-databases}
Modern SQL and NoSQL databases are designed to scale both data and the load of operations accessing that data over many servers that do not share disk or RAM, so-called `shared nothing' architecture \cite{RN67}.  We may partition data {\itshape vertically}, dividing tables into groups of columns that may be placed on different data nodes; or {\itshape horizontally}, where the split is by row \cite{RN68}. 

In the use case, the quantity of data does not approach the levels of `Big Data' applications.  Partitioning is proposed instead as a means of scaling the demand for that data.  The ticketing system will not require a large number of columns and the three operations outlined do not have significantly different column requirements, therefore horizontal partitioning is most relevant.  The partition key of a Ticket table may be the Ticket Type, the Date, or the seat Row.  Demand for tickets is likely to vary by each of these attributes.  An alternative partitioning strategy would be to on denormalised tables supporting the query, book and return operations.  The load on each data node would follow the demand for the data types and operations placed there.

One issue to be aware of is {\itshape replication}.   Most distributed databases offer replication of data from one partition to another for availability.  In the use case, if demand overloads a data node, the database may share the throughput using a copy of the data on another data node.  If this is also the primary data node of an otherwise low demand data type, then it may be overwhelmed in turn i.e. the skewed demand has followed the data.

\begin{figure}
	\label{figure:consistent_hashing}
	\centering
	\includegraphics[trim = 5 5 5 5, clip, width=\textwidth]{img/dbdist}
	\caption{Consistent hashing, without and with replication}
\end{figure}
\FloatBarrier
The Cassandra database has an interesting method of partitioning data, using {\itshape consistent hashing} (also used by Riak, Redis and BigData among others \cite{RN66}).  The largest output of a hash function wraps round to the smallest value so that the range of hash values forms a conceptual `ring'.  Each data node is assigned a position on this ring, then the hash value of the partition key of a data item is used to determine the node used to store it.  When using replication, with a replication factor of {\itshape N}, a copy of the data is placed on the next {\itshape N-1} nodes walking clockwise round the ring \cite{RN1050}.  This is illustrated in Figure \ref{figure:consistent_hashing}.

\section{Modelling}\label{sec:modelling}

The modelling technique must enable predictions about throughput for varying levels of skewed demand.  It must also be possible to compose system models from simpler components.  Two current approaches for modelling cloud computing infrastructures are simulators or mathematical language-based models (e.g. {\itshape Process Algebra}).

\paragraph{Simulation Frameworks.} Several cloud simulation tools have been developed in recent years \cite{RN1081}, enabling the construction of models of Infrastructure as a Service environments.  They are often concerned with modelling the efficient running of that infrastructure, for example power usage, but also include utilisation models and may be useful for predicting the effect of shared demand or resource scheduling algorithms.  CloudSim \cite{RN69} is one such framework, requiring Java development for building the models.  This is an overhead for creation but offers flexibility in use.

\paragraph{Process Algebra.} Process Algebras (such as PEPA or TIPP \cite{RN64}) model throughput in interdependent processes, with a mixture of independent and shared actions operating at different rates.  There is a PEPA Eclipse plugin \cite{RN1080} that allows PEPA specifications to be parsed and run like programs, aiding experimentation on a range of action rates by automating repetitive calculations.

\subsection{PEPA (Performance Evaluation Process Algebra)}

The models will be produced using PEPA.  This paper is concerned with distribution of throughput in complex systems, rather than right-sizing those systems.  The PEPA Workbench will allow the automation of testing with a range of skewed demand values.

A PEPA model describes a system of interacting {\itshape components} which carry out {\itshape activities} at specified or passive {\itshape rates}.  A component is usually denoted by a name with an initial upper case letter, e.g. {\itshape Website}, and an activity type and rate are expressed as a bracketed pair e.g. $\mathit(request, r)$ where the activity type is normally a full lower case name and the rate is a single letter or the top symbol $\top$, denoting an unspecified (passive) rate.  There is a set of combinators that describe how the components and activities interact.  This paper uses the following subset, for the full syntax see {\cite{RN1051}}:

\subsubsection{Prefix:} $(\mathit{\alpha},\mathit{r}).\mathit{P}$ - a component carries out activity $\mathit{\alpha}$ at rate $\mathit{r}$ and then behaves as component $\mathit{P}$.
\subsubsection{Constant:} $\mathit{A} \rmdef \mathit{P}$ - assign the behaviour of component $\mathit{P}$ to the constant $\mathit{A}$.  Used with prefix, this can be used to define a recurring process e.g. $\mathit{P} \rmdef (\mathit{\alpha},\mathit{r}).\mathit{P}$.
\subsubsection{Choice:} $\mathit{P} + \mathit{Q}$ - a component may behave {\itshape either} as component $\mathit{P}$ or $\mathit{Q}$, non-deterministically.  This represents a race condition between components.
\subsubsection{Cooperation:} $\mathit{P} \sync{L} \mathit{Q}$ - for shared activities in the set $\mathit{L}$, components $\mathit{P}$ and $\mathit{Q}$ may only proceed with the simultaneous execution of those activities at the rate of the slowest component, otherwise they behave independently.
\subsubsection{Parallel:} $\mathit{P} \parallel \mathit{Q}$ - shorthand for components that synchronize with no shared activities i.e. equivalent to $\mathit{P} \sync{\emptyset} \mathit{Q}$.
\subsubsection{Aggregation:} $\mathit{P}[N]$ - represents $\mathit{N}$ instances of component $\mathit{P}$, but does not distinguish which instance of $\mathit{P}$ changes.  So for example where some component has states $\mathit{P1}$ and $\mathit{P2}$, and $\mathit{(P1|P2)}$ does not equal $\mathit{(P2|P1)}$, this model has 4 states.  If it doesn't matter which component has changed, then the model has only 3 states and can be written as $\mathit{P}[2]$.

\paragraph{Solutions.} PEPA models are used to represent a system using a stochastic process, where the activity durations are random variables.  Where these are negative exponentially distributed, then this representation is a continuous time Markov process with a steady state solution over a period of time \cite{RN1051}.  This may be calculated using the PEPA Eclipse plugin.

%
% ---- Model components sub-document
%

%
% ---- PEPA Component Models
%

\section{PEPA Component Models}

%
% ---- Distributed database without replication
%
\subsection{Distributed database without replication}

The PEPA model for a distributed database is shown in Figure \ref{figure:pepa_dd_model}.

\begin{figure}
	\caption{Distributed database PEPA model}
	\label{figure:pepa_dd_model}
	\centering
% Automatically generated by PEPA2Latex
% --begin
\begin{displaymath}
	\begin{array}{rcl}
		\mathit{a} & = & 1.0-10.0\\
		\mathit{c} & = & 1.0\\
		\mathit{db} & = & 5.0\\
		[0.0ex]		\mathit{Website} & \rmdef & (\mathit{book_a},\mathit{a}).\mathit{Website}+(\mathit{book_c},\mathit{c}).\mathit{Website}\\
		\mathit{DB_1} & \rmdef & (\mathit{book_a},\top).\mathit{DBsrv_1}\\
		\mathit{DBsrv_1} & \rmdef & (\mathit{dbsrv_1},\top).\mathit{DB_1}\\
		\mathit{DB_2} & \rmdef & (\mathit{book_c},\top).\mathit{DBsrv_2}\\
		\mathit{DBsrv_2} & \rmdef & (\mathit{dbsrv_2},\top).\mathit{DB_2}\\
		\mathit{Service_1} & \rmdef & (\mathit{dbsrv_1},\mathit{db}).\mathit{Service_1}\\
		\mathit{Service_2} & \rmdef & (\mathit{dbsrv_2},\mathit{db}).\mathit{Service_2}\\
		[0.0ex]		\multicolumn{3}{l}{\mathit{Website}\sync{book_a,book_c}\mathit{DB_1}\parallel\mathit{DB_2}\sync{dbsrv_1,dbsrv_2}\mathit{Service_1}\parallel\mathit{Service_2}}\\
		[0.0ex]	\end{array}
\end{displaymath}
% --end
\end{figure}

See the experimental results in Table \ref{table:dd_results}.

\begin{table}[h!]
	\begin{center}
		\caption{Distributed database experimental results}
		\label{table:dd_results}
			\pgfplotstabletypeset[
			col sep=comma,
			string type,
			columns/a/.style={column name=a, column type={p{.1\textwidth}}},
			columns/booka/.style={column name=book\textsubscript{a}, column type={p{.1\textwidth}}},
			columns/bookc/.style={column name=book\textsubscript{c}, column type={p{.1\textwidth}}},
			columns/dbsrv1/.style={column name=dbsrv\textsubscript{1}, column type={p{.1\textwidth}}},
			columns/dbsrv2/.style={column name=dbsrv\textsubscript{2}, column type={p{.1\textwidth}}},
			every head row/.style={before row=\hline Rate & \multicolumn{4}{c}{Throughput} \\,after row=\hline},
			every last row/.style={after row=\hline},
			]{data/ddnr/results.csv}
	\end{center}
\end{table}

\begin{figure}
	\caption{Distributed database experimental results}
	\label{figure:dd_charts}
	\centering
	\begin{tikzpicture}
	\begin{axis}[
		title={Throughput against input rate a},
		xlabel={Rate a},
		ylabel={Throughput},
		xmin=0, xmax=10,
		ymin=0, ymax=5,
		legend pos=north west,
		ymajorgrids=true,
		grid style=dashed,
		cycle multiindex* list={
			mark list*
				\nextlist
			cyan,brown,green,blue,red
		}
	]
	
	\addplot table [x index={0}, y index={1}, col sep=comma]{data/ddnr/book_a.csv};
	\addplot table [x index={0}, y index={1}, col sep=comma]{data/ddnr/book_c.csv};

	\legend{a,c}

	\end{axis}
	\end{tikzpicture}

	\begin{tikzpicture}
	\begin{axis}[
		title={Throughput against input rate a},
		xlabel={Rate a},
		ylabel={Throughput},
		xmin=0, xmax=10,
		ymin=0, ymax=5,
		legend pos=north west,
		ymajorgrids=true,
		grid style=dashed,
		cycle multiindex* list={
			mark list*
				\nextlist
			cyan,brown,green,blue,red
		}
	]
	
	\addplot table [x index={0}, y index={1}, col sep=comma]{data/ddnr/dbsrv_1.csv};
	\addplot table [x index={0}, y index={1}, col sep=comma]{data/ddnr/dbsrv_2.csv};

	\legend{dbsrv\textsubscript{1},dbsrv\textsubscript{2}}

	\end{axis}
	\end{tikzpicture}
\end{figure}

%
% ---- Distributed database with replication
%
\subsection{Distributed database with replication}

\begin{figure}
	\caption{Distributed database with replication PEPA model}
	\label{figure:pepa_ddwr_model}
	\centering
% Automatically generated by PEPA2Latex
% --begin
\begin{displaymath}
	\begin{array}{rcl}
		\mathit{a} & = & 1.0-10.0\\
		\mathit{c} & = & 1.0\\
		\mathit{d} & = & 1.0\\
		\mathit{db} & = & 5.0\\
		[0.0ex]		\mathit{Website} & \rmdef & (\mathit{book_a},\mathit{a}).\mathit{Website}+(\mathit{book_c},\mathit{c}).\mathit{Website}+(\mathit{book_d},\mathit{d}).\mathit{Website}\\
		\mathit{DB_1} & \rmdef & (\mathit{book_a},\top).\mathit{DBsrv_1}+(\mathit{book_c},\top).\mathit{DBsrv_1}\\
		\mathit{DBsrv_1} & \rmdef & (\mathit{dbsrv_1},\top).\mathit{DB_1}\\
		\mathit{DB_2} & \rmdef & (\mathit{book_c},\top).\mathit{DBsrv_2}+(\mathit{book_d},\top).\mathit{DBsrv_2}\\
		\mathit{DBsrv_2} & \rmdef & (\mathit{dbsrv_2},\top).\mathit{DB_2}\\
		\mathit{DB_3} & \rmdef & (\mathit{book_d},\top).\mathit{DBsrv_3}+(\mathit{book_a},\top).\mathit{DBsrv_3}\\
		\mathit{DBsrv_3} & \rmdef & (\mathit{dbsrv_3},\top).\mathit{DB_3}\\
		\mathit{Service_1} & \rmdef & (\mathit{dbsrv_1},\mathit{db}).\mathit{Service_1}\\
		\mathit{Service_2} & \rmdef & (\mathit{dbsrv_2},\mathit{db}).\mathit{Service_2}\\
		\mathit{Service_3} & \rmdef & (\mathit{dbsrv_3},\mathit{db}).\mathit{Service_3}\\
		[0.0ex]		\multicolumn{3}{l}{\mathit{Website}\sync{book_a,book_c,book_d}\mathit{DB_1}\parallel\mathit{DB_2}\parallel\mathit{DB_3}\sync{dbsrv_1,dbsrv_2,dbsrv_3}\mathit{Service_1}\parallel\mathit{Service_2}\parallel\mathit{Service_3}}\\
		[0.0ex]	\end{array}
\end{displaymath}
% --end
\end{figure}

See the experimental results in Table \ref{table:ddwr_results}.

\begin{table}[h!]
	\begin{center}
		\caption{Distributed database with replication experimental results}
		\label{table:ddwr_results}
		\pgfplotstabletypeset[
		col sep=comma,
		string type,
		columns/a/.style={column name=a, column type={p{.1\textwidth}}},
		columns/booka/.style={column name=book\textsubscript{a}, column type={p{.1\textwidth}}},
		columns/bookc/.style={column name=book\textsubscript{c}, column type={p{.1\textwidth}}},
		columns/bookd/.style={column name=book\textsubscript{d}, column type={p{.1\textwidth}}},
		columns/dbsrv1/.style={column name=dbsrv\textsubscript{1}, column type={p{.1\textwidth}}},
		columns/dbsrv2/.style={column name=dbsrv\textsubscript{2}, column type={p{.1\textwidth}}},
		columns/dbsrv3/.style={column name=dbsrv\textsubscript{3}, column type={p{.1\textwidth}}},
		every head row/.style={before row=\hline Rate & \multicolumn{6}{c}{Throughput} \\,after row=\hline},
		every last row/.style={after row=\hline},
		]{data/ddwr/results.csv}
	\end{center}
\end{table}

\begin{figure}
	\caption{Distributed database with replication experimental results}
	\label{figure:ddwr_charts}
	\centering
	\begin{tikzpicture}
	\begin{axis}[
		title={Throughput against input rate a},
		width=0.9\textwidth,
		ybar=0pt,
		bar width=.02\textwidth,
		xlabel={Rate a},
		ylabel={Throughput},
		xtick=data,
		ymin=0, ymax=6,
		legend pos=north west,
		ymajorgrids=true,
		grid style=dashed,
	]
	
	\addplot [pattern=north east lines, pattern color=blue] table [x index={0}, y index={1}, col sep=comma]{data/ddwr/book_a.csv};
	\addplot [fill=green] table [x index={0}, y index={1}, col sep=comma]{data/ddwr/book_c.csv};
	\addplot [pattern=crosshatch, pattern color=brown] table [x index={0}, y index={1}, col sep=comma]{data/ddwr/book_d.csv};

	\legend{a,c,d}

	\end{axis}
	\end{tikzpicture}

	\begin{tikzpicture}
	\begin{axis}[
		title={Throughput against input rate a},
		width=0.9\textwidth,
		ybar=0pt,
		bar width=.02\textwidth,
		xlabel={Rate a},
		ylabel={Throughput},
		xtick=data,
		ymin=0, ymax=6,
		legend pos=north west,
		ymajorgrids=true,
		grid style=dashed,
	]
	
	\addplot [pattern=north east lines, pattern color=blue] table [x index={0}, y index={1}, col sep=comma]{data/ddwr/dbsrv_1.csv};
	\addplot [fill=green] table [x index={0}, y index={1}, col sep=comma]{data/ddwr/dbsrv_2.csv};
	\addplot [pattern=crosshatch, pattern color=brown] table [x index={0}, y index={1}, col sep=comma]{data/ddwr/dbsrv_3.csv};

	\legend{dbsrv\textsubscript{1},dbsrv\textsubscript{2},dbsrv\textsubscript{3}}

	\end{axis}
	\end{tikzpicture}
\end{figure}

%
% ---- Shared middleware queue
%
\subsection{Shared middleware queue}

Queues have already been extensively modelled in PEPA \cite{RN75}...

\begin{figure}
	\caption{Generic shared queue PEPA model}
	\label{figure:pepa_queue_model}
	\centering
	% Automatically generated by PEPA2Latex
	% --begin
	\begin{displaymath}
		\begin{array}{rcl}
			\mathit{a_1} & = & 1.0-10.0\\
			\mathit{s_1} & = & 5.0\\
			\mathit{a_2} & = & 1.0\\
			\mathit{s_2} & = & 5.0\\
			[0.0ex]		\mathit{Arrival_1} & \rmdef & (\mathit{arrive_1},\mathit{a_1}).\mathit{Arrival_1}\\
			\mathit{Service_1} & \rmdef & (\mathit{serve_1},\mathit{s_1}).\mathit{Service_1}\\
			\mathit{Arrival_2} & \rmdef & (\mathit{arrive_2},\mathit{a_2}).\mathit{Arrival_2}\\
			\mathit{Service_2} & \rmdef & (\mathit{serve_2},\mathit{s_2}).\mathit{Service_2}\\
			\mathit{Q_0} & \rmdef & (\mathit{arrive_1},\top).\mathit{Q_1}+(\mathit{arrive_2},\top).\mathit{Q_2}\\
			\mathit{Q_1} & \rmdef & (\mathit{serve_1},\top).\mathit{Q_0}\\
			\mathit{Q_2} & \rmdef & (\mathit{serve_2},\top).\mathit{Q_0}\\
			[0.0ex]		\multicolumn{3}{l}{\mathit{Arrival_1}\sync{arrive_1}\mathit{Q_0}[N]\sync{serve_1}\mathit{Service_1}\sync{arrive_2}\mathit{Arrival_2}\sync{serve_2}\mathit{Service_2}}\\
			[0.0ex]	\end{array}
	\end{displaymath}
	% --end
\end{figure}

\begin{figure}
	\caption{Generic shared queue experimental results}
	\label{figure:queue_charts}
	\centering
	\begin{tikzpicture}
	\begin{axis}[
		title={Throughput of arrive\textsubscript{1} against input rate a\textsubscript{1} for different queue lengths N},
		xlabel={Rate a\textsubscript{1}},
		ylabel={Throughput arrive\textsubscript{1}},
		xmin=0, xmax=10,
		ymin=0, ymax=5,
		legend pos=north west,
		ymajorgrids=true,
		grid style=dashed,
		cycle multiindex* list={
			mark list*
				\nextlist
			cyan,brown,green,blue,red
		}
	]
	
	\addplot table [x index={0}, y index={1}, col sep=comma]{data/queue/N1_arrive_1.csv};
	\addplot table [x index={0}, y index={1}, col sep=comma]{data/queue/N2_arrive_1.csv};
	\addplot table [x index={0}, y index={1}, col sep=comma]{data/queue/N5_arrive_1.csv};
	\addplot table [x index={0}, y index={1}, col sep=comma]{data/queue/N10_arrive_1.csv};
	\addplot table [x index={0}, y index={1}, col sep=comma]{data/queue/N20_arrive_1.csv};
	
	\legend{N = 1, N = 2, N = 5, N = 10, N = 20}

	\end{axis}
	\end{tikzpicture}
	
	\begin{tikzpicture}
	\begin{axis}[
	title={Throughput of serve\textsubscript{1} against input rate a\textsubscript{1} for different queue lengths N},
	xlabel={Rate a\textsubscript{1}},
	ylabel={Throughput serve\textsubscript{1}},
	xmin=0, xmax=10,
	ymin=0, ymax=5,
	legend pos=north west,
	ymajorgrids=true,
	grid style=dashed,
	cycle multiindex* list={
		mark list*
		\nextlist
		cyan,brown,green,blue,red
	}
	]
	
	\addplot table [x index={0}, y index={1}, col sep=comma]{data/queue/N1_serve_1.csv};
	\addplot table [x index={0}, y index={1}, col sep=comma]{data/queue/N2_serve_1.csv};
	\addplot table [x index={0}, y index={1}, col sep=comma]{data/queue/N5_serve_1.csv};
	\addplot table [x index={0}, y index={1}, col sep=comma]{data/queue/N10_serve_1.csv};
	\addplot table [x index={0}, y index={1}, col sep=comma]{data/queue/N20_serve_1.csv};
	
	\legend{N = 1, N = 2, N = 5, N = 10, N = 20}
	
	\end{axis}
	\end{tikzpicture}
	
	\begin{tikzpicture}
	\begin{axis}[
	title={Throughput of arrive\textsubscript{2} against input rate a\textsubscript{1} for different queue lengths N},
	xlabel={Rate a\textsubscript{1}},
	ylabel={Throughput arrive\textsubscript{2}},
	xmin=0, xmax=10,
	ymin=0, ymax=1,
	legend pos=south west,
	ymajorgrids=true,
	grid style=dashed,
	cycle multiindex* list={
		mark list*
		\nextlist
		cyan,brown,green,blue,red
	}
	]
	
	\addplot table [x index={0}, y index={1}, col sep=comma]{data/queue/N1_arrive_2.csv};
	\addplot table [x index={0}, y index={1}, col sep=comma]{data/queue/N2_arrive_2.csv};
	\addplot table [x index={0}, y index={1}, col sep=comma]{data/queue/N5_arrive_2.csv};
	\addplot table [x index={0}, y index={1}, col sep=comma]{data/queue/N10_arrive_2.csv};
	\addplot table [x index={0}, y index={1}, col sep=comma]{data/queue/N20_arrive_2.csv};
	
	\legend{N = 1, N = 2, N = 5, N = 10, N = 20}
	
	\end{axis}
	\end{tikzpicture}
\end{figure}

%
% ---- Model system sub-document
%

%
% ---- PEPA System Models
%

\section{PEPA System Models}\label{sec:pepa-system-models}

The components are combined into models of full distributed architectures, that may be implemented and tested as working built systems.  There are three system models - a simplified microservices architecture, and two models composed of a shared queue and distributed database, with and without replication.

%
% ---- Simple microservices
%
\FloatBarrier
\subsection{Simple microservices}

The simple microservices model has separate end-to-end services for handling athletics and cycling ticket requests.  This is not a `natural' microservices implementation, which would be more likely to separate on operations e.g. searching, booking and returning tickets.  Choosing this design however makes the system directly comparable to the partitioning strategy used for the distributed database models.  It is not in itself functionally different to separating services by operations unless considering additional features to handle eventual data consistency (see Future Work).

The system has two separate databases, one for athletics tickets, one for cycling, and each has its own dedicated worker application (Figure \ref{figure:simplemicro_architecture}).  The PEPA model of this (Figure \ref{figure:simplemicro}) uses the same database process building blocks as the distributed database component models, but in this case they cooperate with the dedicated worker application processes $\mathit{Worker_A}$ and $\mathit{Worker_C}$.

As for the component models, there are two arrival processes, dealing with cycling requests at the rate $\mathit{c=1.0}$ and athletics requests at rates 1.0 to 10.0 in steps of 1.0.  Again each database may serve requests at a maximum rate of {\itshape db}.  Note that this rate has been changed to 6.5 so that it is proportional to the performance observed when testing the built system (the model has been tuned following measurement of the system).

The worker application processes have a maximum rate of $\mathit{w=100.0}$.  This value has been chosen to be much higher than the other parts of the system to minimise its impact on the system testing.  The assumption being made here is that the applications may be designed to cope with this much higher demand, perhaps using Elastic Scaling of servers, which is out of scope of this paper.

\begin{figure}
	\caption{Simple microservices architecture}
	\label{figure:simplemicro_architecture}
	\centering
	\includegraphics[trim = 5 5 5 5, clip, width=\textwidth]{img/simplemicro}
\end{figure}

\begin{figure}
	\caption{Simple microservices PEPA model}
	\label{figure:simplemicro}
	\centering
	% Automatically generated by PEPA2Latex
	% --begin
	\begin{displaymath}
		\begin{array}{rcl}
			\mathit{a} & = & 1.0 - 10.0\\
			\mathit{c} & = & 1.0\\
			\mathit{w} & = & 100.0\\
			\mathit{db} & = & 6.5\\
			[2.0ex]		\mathit{Website} & \rmdef & (\mathit{athletics},\mathit{a}).\mathit{Website}+(\mathit{cycling},\mathit{c}).\mathit{Website}\\
			\mathit{Worker_A} & \rmdef & (\mathit{athletics},\top).\mathit{WorkerSrv_A}\\
			\mathit{WorkerSrv_A} & \rmdef & (\mathit{workerA},\top).\mathit{Worker_A}\\
			\mathit{Worker_C} & \rmdef & (\mathit{cycling},\top).\mathit{WorkerSrv_C}\\
			\mathit{WorkerSrv_C} & \rmdef & (\mathit{workerC},\top).\mathit{Worker_C}\\
			\mathit{DB_1} & \rmdef & (\mathit{workerA},\mathit{w}).\mathit{DBsrv_1}\\
			\mathit{DBsrv_1} & \rmdef & (\mathit{dbsrv1},\mathit{db}).\mathit{DB_1}\\
			\mathit{DB_2} & \rmdef & (\mathit{workerC},\mathit{w}).\mathit{DBsrv_2}\\
			\mathit{DBsrv_2} & \rmdef & (\mathit{dbsrv2},\mathit{db}).\mathit{DB_2}\\
			\mathit{Service_1} & \rmdef & (\mathit{dbsrv1},\mathit{db}).\mathit{Service_1}\\
			\mathit{Service_2} & \rmdef & (\mathit{dbsrv2},\mathit{db}).\mathit{Service_2}\\
			[2.0ex]		\multicolumn{3}{l}{\mathit{Service_1}\sync{dbsrv1}\mathit{DB_1}\sync{workerA}\mathit{Worker_A}\sync{athletics}\mathit{Website}\sync{cycling}\mathit{Worker_C}\sync{workerC}\mathit{DB_2}\sync{dbsrv2}\mathit{Service_2}}\\
			[2.0ex]	\end{array}
	\end{displaymath}
	% --end
\end{figure}

The PEPA Eclipse plugin experiments test each input rate {\itshape a} of athletics requests from 1.0 to 10.0 in steps of 1.0, with all other rates fixed.  Table \ref{table:micro_results} shows the numerical results, and Figure \ref{figure:micro_charts} shows the throughput of {\itshape athletics} and {\itshape cycling} against input rate {\itshape a}.
These demonstrate that:
\begin{itemize}
	\item the throughput of {\itshape athletics} is constrained by the maximum service rate of the database handling those requests. Both athletics and cycling activities demonstrate some loss of throughput, though less than with the distributed database component (perhaps due to the additional worker application processes producing a partial decoupling effect).
	\item the throughput of {\itshape cycling} is independent of athletics.  This supports the claim that microservices architecture isolates the skewed demand.
	\item the database node throughput follows the throughput of each sport activity.
\end{itemize}

\begin{table}[h!]
	\begin{center}
		\caption{Simple microservices experimental results}
		\label{table:micro_results}
		\pgfplotstabletypeset[
		col sep=comma,
		string type,
		columns/a/.style={column name=a, column type={p{.1\textwidth}}},
		columns/athletics/.style={column name=athletics, column type={p{.1\textwidth}}},
		columns/cycling/.style={column name=cycling, column type={p{.1\textwidth}}},
		columns/workerA/.style={column name=workerA, column type={p{.1\textwidth}}},
		columns/workerC/.style={column name=workerC, column type={p{.1\textwidth}}},
		columns/dbsrv1/.style={column name=dbsrv1, column type={p{.1\textwidth}}},
		columns/dbsrv2/.style={column name=dbsrv2, column type={p{.1\textwidth}}},
		every head row/.style={before row=\hline Rate & \multicolumn{6}{c}{Throughput} \\,after row=\hline},
		every last row/.style={after row=\hline},
		]{data/micro/results.csv}
	\end{center}
\end{table}

\begin{figure}
	\caption{Simple microservices experimental results}
	\label{figure:micro_charts}
	\centering
	\begin{tikzpicture}
	\begin{axis}[
	title={Throughput against input rate a},
	xlabel={Rate a},
	ylabel={Throughput},
	xmin=0, xmax=10,
	ymin=0, ymax=7,
	legend pos=north west,
	ymajorgrids=true,
	grid style=dashed,
	cycle multiindex* list={
		mark list*
		\nextlist
		cyan,brown,green,blue,red
	}
	]
	
	\addplot table [x index={0}, y index={1}, col sep=comma]{data/micro/athletics.csv};
	\addplot table [x index={0}, y index={1}, col sep=comma]{data/micro/cycling.csv};
	
	\legend{athletics,cycling}
	
	\end{axis}
	\end{tikzpicture}
\end{figure}

%
% ---- Shared queue and distributed database
%
\FloatBarrier
\subsection{Shared queue and distributed database}

The next system model (Figure \ref{figure:queuedd_architecture}) is the first combination of the shared queue and distributed database components.  A website sends all ticket requests asynchronously to a cloud queue service, and one or more worker applications (either a multi-threaded application or a scaling set of applications) dequeues the requests and forwards them to the distributed database.  The database uses a horizontal partitioning strategy based on the sport, without any replication.

The PEPA model (Figure \ref{figure:queuedd}) is a straightforward combination of the shared queue and distributed database component models.  A queue length of N=10 is used as the experiments showed that for a small state space, this allowed the athletics throughput to get very close to the maximum service rate.

There is no separate representation of worker processes (the previous system model showed that the throughputs of the worker activities were exactly the same as the processes on either side of them) but as before a high maximum rate has specified $\mathit{q=100.0}$ for the rate at which requests may be dequeued.  This is the rate used for {\itshape queueA} and {\itshape queueB} requests arriving at the database processes.

\begin{figure}
	\caption{Shared queue middleware architecture}
	\label{figure:queuedd_architecture}
	\centering
	\includegraphics[trim = 5 5 5 5, clip, width=\textwidth]{img/sharedqueue}
\end{figure}

\begin{figure}
	\caption{Shared queue and distributed database}
	\label{figure:queuedd}
	\centering
	% Automatically generated by PEPA2Latex
	% --begin
	\begin{displaymath}
	\begin{array}{rcl}
	\mathit{a} & = & 1.0\\
	\mathit{c} & = & 1.0\\
	\mathit{q} & = & 100.0\\
	\mathit{db} & = & 5.0\\
	[2.0ex]		\mathit{Website} & \rmdef & (\mathit{athletics},\mathit{a}).\mathit{Website}+(\mathit{cycling},\mathit{c}).\mathit{Website}\\
	\mathit{Q_0} & \rmdef & (\mathit{athletics},\top).\mathit{Q_A}+(\mathit{cycling},\top).\mathit{Q_C}\\
	\mathit{Q_A} & \rmdef & (\mathit{queueA},\top).\mathit{Q_0}\\
	\mathit{Q_C} & \rmdef & (\mathit{queueC},\top).\mathit{Q_0}\\
	[2.0ex]		\mathit{DB_1} & \rmdef & (\mathit{queueA},\mathit{q}).\mathit{DBsrv_1}\\
	\mathit{DBsrv_1} & \rmdef & (\mathit{dbsrv1},\mathit{db}).\mathit{DB_1}\\
	\mathit{DB_2} & \rmdef & (\mathit{queueC},\mathit{q}).\mathit{DBsrv_2}\\
	\mathit{DBsrv_2} & \rmdef & (\mathit{dbsrv2},\mathit{db}).\mathit{DB_2}\\
	\mathit{Service_1} & \rmdef & (\mathit{dbsrv1},\mathit{db}).\mathit{Service_1}\\
	\mathit{Service_2} & \rmdef & (\mathit{dbsrv2},\mathit{db}).\mathit{Service_2}\\
	[2.0ex]		\multicolumn{3}{l}{\mathit{Website}\sync{\substack{athletics\\cycling}}\mathit{Q_0}[10]\sync{\substack{queueA\\queueC}}\mathit{DB_1}\parallel\mathit{DB_2}\sync{\substack{dbsrv1\\dbsrv2}}\mathit{Service_1}\parallel\mathit{Service_2}}\\
	[2.0ex]	\end{array}
	\end{displaymath}
	% --end
\end{figure}

As is now usual, the Eclipse plugin is used to find the steady-state throughputs of the activities for input rates of athletics requests increasing from 1.0 to 10.0 with cycling and other rates fixed.  The results appear numerically in Table \ref{table:queuedd_results} and as a chart comparing {\itshape athletics} and {\itshape cycling} throughput in Figure \ref{figure:queueddnr_sport}.
They show:
\begin{itemize}
	\item the throughput of {\itshape athletics} is constrained by the database service rate of a single node.  Athletics and cycling activities demonstrate loss of throughput, but less than shown for the distributed database component model.  This may indicate the decoupling effect of the middleware queue.	
	\item the throughput of {\itshape cycling} is constrained by the ratio between the input rates of athletics and cycling, as it was with the shared queue component.  The behaviour of the queue appears to be the most significant when combined into a system.
	\item the database node throughput follows the throughput of each sport activity, i.e. the partitioning strategy routes all of the demand onto the node handling that sport.
\end{itemize}

\begin{table}[h!]
	\begin{center}
		\caption{Shared queue and distributed database experimental results}
		\label{table:queuedd_results}
		\pgfplotstabletypeset[
		col sep=comma,
		string type,
		columns/a/.style={column name=a, column type={p{.1\textwidth}}},
		columns/athletics/.style={column name=athletics, column type={p{.1\textwidth}}},
		columns/cycling/.style={column name=cycling, column type={p{.1\textwidth}}},
		columns/ratio/.style={column name=ratio, column type={p{.1\textwidth}}},
		columns/qa/.style={column name=queueA, column type={p{.1\textwidth}}},
		columns/qc/.style={column name=queueC, column type={p{.1\textwidth}}},
		columns/dbsrv1/.style={column name=dbsrv1, column type={p{.1\textwidth}}},
		columns/dbsrv2/.style={column name=dbsrv2, column type={p{.1\textwidth}}},
		every head row/.style={before row=\hline Rate & \multicolumn{7}{c}{Throughput} \\,after row=\hline},
		every last row/.style={after row=\hline},
		]{data/qddnr/results.csv}
	\end{center}
\end{table}

\begin{figure}
	\caption{Shared queue and distributed database - sport throughput}
	\label{figure:queueddnr_sport}
	\centering
	\begin{tikzpicture}
	\begin{axis}[
	title={Throughput against input rate a},
	xlabel={Rate a},
	ylabel={Throughput},
	xmin=0, xmax=10,
	ymin=0, ymax=5,
	legend pos=north west,
	ymajorgrids=true,
	grid style=dashed,
	cycle multiindex* list={
		mark list*
		\nextlist
		cyan,brown,green,blue,red
	}
	]
	
	\addplot table [x index={0}, y index={1}, col sep=comma]{data/qddnr/athletics.csv};
	\addplot table [x index={0}, y index={1}, col sep=comma]{data/qddnr/cycling.csv};
	
	\legend{athletics,cycling}
	
	\end{axis}
	\end{tikzpicture}
\end{figure}

%
% ---- Shared queue and distributed database with replication
%
\FloatBarrier
\subsection{Shared queue and distributed database with replication}

The final system model  (Figure \ref{figure:queueddrep_architecture}) combines a shared queue with a distributed database using replication, with one replica of each data partition on the `next' node using consistent hashing.  As before the website sends ticket requests via a shared queue to a worker application, which dequeues them for the database.

The PEPA model (Figure \ref{figure:queueddrep}) combines the shared queue and distributed database with replication components, so that there is now an additional queue state $\mathit{Q_D}$ for holding {\itshape diving} ticket requests, and there are three data node processes.  $\mathit{DB_1}$ is able to service {\itshape athletics} and {\itshape cycling} requests,  $\mathit{DB_2}$ handles {\itshape cycling} and {\itshape diving}, and $\mathit{DB_3}$ handles {\itshape athletics} and {\itshape diving}.  As with the component node, the model's assumption is that each data node handles each supported sport with equal probability.  Again the model uses a queue length of N=10 and maximum queue worker rate of $\mathit{q=100.0}$.

\begin{figure}
	\caption{Distributed database with replication architecture}
	\label{figure:queueddrep_architecture}
	\centering
	\includegraphics[trim = 5 5 5 5, clip, width=\textwidth]{img/sharedqueue_withrep}
\end{figure}

\begin{figure}
	\caption{Shared queue and distributed database with replication}
	\label{figure:queueddrep}
	\centering
	% Automatically generated by PEPA2Latex
	% --begin
	\begin{displaymath}
	\begin{array}{rcl}
	\mathit{a} & = & 1.0\\
	\mathit{c} & = & 1.0\\
	\mathit{d} & = & 1.0\\
	\mathit{q} & = & 100.0\\
	\mathit{db} & = & 5.0\\
	[2.0ex]		\mathit{Website} & \rmdef & (\mathit{athletics},\mathit{a}).\mathit{Website}+(\mathit{cycling},\mathit{c}).\mathit{Website}+(\mathit{diving},\mathit{d}).\mathit{Website}\\
	\mathit{Q_0} & \rmdef & (\mathit{athletics},\top).\mathit{Q_A}+(\mathit{cycling},\top).\mathit{Q_C}+(\mathit{diving},\top).\mathit{Q_D}\\
	\mathit{Q_A} & \rmdef & (\mathit{queueA},\top).\mathit{Q_0}\\
	\mathit{Q_C} & \rmdef & (\mathit{queueC},\top).\mathit{Q_0}\\
	\mathit{Q_D} & \rmdef & (\mathit{queueD},\top).\mathit{Q_0}\\
	[2.0ex]		\mathit{DB_1} & \rmdef & (\mathit{queueA},\mathit{q}).\mathit{DBsrv_1}+(\mathit{queueC},\mathit{q}).\mathit{DBsrv_1}\\
	\mathit{DBsrv_1} & \rmdef & (\mathit{dbsrv1},\top).\mathit{DB_1}\\
	\mathit{DB_2} & \rmdef & (\mathit{queueC},\mathit{q}).\mathit{DBsrv_2}+(\mathit{queueD},\mathit{q}).\mathit{DBsrv_2}\\
	\mathit{DBsrv_2} & \rmdef & (\mathit{dbsrv2},\top).\mathit{DB_2}\\
	\mathit{DB_3} & \rmdef & (\mathit{queueD},\mathit{q}).\mathit{DBsrv_3}+(\mathit{queueA},\mathit{q}).\mathit{DBsrv_3}\\
	\mathit{DBsrv_3} & \rmdef & (\mathit{dbsrv3},\top).\mathit{DB_3}\\
	\mathit{Service_1} & \rmdef & (\mathit{dbsrv1},\mathit{db}).\mathit{Service_1}\\
	\mathit{Service_2} & \rmdef & (\mathit{dbsrv2},\mathit{db}).\mathit{Service_2}\\
	\mathit{Service_3} & \rmdef & (\mathit{dbsrv3},\mathit{db}).\mathit{Service_3}\\
	[2.0ex]		\multicolumn{3}{l}{\mathit{Website}\sync{\substack{athletics\\cycling\\diving}}\mathit{Q_0}[10]\sync{\substack{queueA\\queueC\\queueD}}\mathit{DB_1}\parallel\mathit{DB_2}\parallel\mathit{DB_3}\sync{\substack{dbsrv1\\dbsrv2\\dbsrv3}}\mathit{Service_1}\parallel\mathit{Service_2}\parallel\mathit{Service_3}}\\
	[2.0ex]	\end{array}
	\end{displaymath}
	% --end
\end{figure}

\FloatBarrier
Experiments are performed in the Eclipse plugin with the usual input rates.  The resulting steady state throughputs are shown in Table \ref{table:queueddwr_results}, in Figure \ref{figure:queueddwr_sport} as a chart comparing {\itshape athletics} and {\itshape cycling} throughput (as the cycling and diving throughputs are identical), and in Figure \ref{figure:queueddwr_database} showing the throughput of the database nodes.
They show:
\begin{itemize}
	\item the throughput of {\itshape athletics} is still constrained but is greater than that of a single database node.  The demand is shared between both data nodes supporting athletics requests.
	\item the throughput of {\itshape cycling} (and diving) is constrained by the ratio between the input rates of athletics and cycling.  In the component model, {\itshape cycling} and {\itshape diving} were impacted by athletics as they both shared a node with athletics tickets.  In the system model, the queue effect appears to outweigh this.
	\item the database node throughput of the two nodes supporting athletics requests is equal, and both are higher than the remaining data node that supports only cycling and diving.  The throughput of this node increases with athletics throughput up to a point, suggesting the node is handling an increasing proportion of the demand for cycling and diving tickets (but that this demand becomes constrained by the queue effect).
	
\end{itemize}

\begin{table}[h!]
	\begin{center}
		\caption{Shared queue and distributed database with replication experimental results}
		\label{table:queueddwr_results}
		\pgfplotstabletypeset[
		col sep=comma,
		string type,
		columns/a/.style={column name=a, column type={p{.07\textwidth}}},
		columns/athletics/.style={column name=athletics, column type={p{.1\textwidth}}},
		columns/cycling/.style={column name=cycling, column type={p{.1\textwidth}}},
		columns/diving/.style={column name=diving, column type={p{.1\textwidth}}},
		columns/ratio/.style={column name=ratio, column type={p{.07\textwidth}}},
		columns/qa/.style={column name=queueA, column type={p{.09\textwidth}}},
		columns/qc/.style={column name=queueC, column type={p{.09\textwidth}}},
		columns/qd/.style={column name=queueD, column type={p{.09\textwidth}}},
		columns/dbsrv1/.style={column name=dbsrv1, column type={p{.09\textwidth}}},
		columns/dbsrv2/.style={column name=dbsrv2, column type={p{.09\textwidth}}},
		columns/dbsrv3/.style={column name=dbsrv3, column type={p{.09\textwidth}}},
		every head row/.style={before row=\hline Rate & \multicolumn{10}{c}{Throughput} \\,after row=\hline},
		every last row/.style={after row=\hline},
		]{data/qddwr/results.csv}
	\end{center}
\end{table}

\begin{figure}
	\caption{Shared queue and distributed database with replication - sport throughput}
	\label{figure:queueddwr_sport}
	\centering
	\begin{tikzpicture}
	\begin{axis}[
	title={Throughput against input rate a},
	xlabel={Rate a},
	ylabel={Throughput},
	xmin=0, xmax=10,
	ymin=0, ymax=9,
	legend pos=north west,
	ymajorgrids=true,
	grid style=dashed,
	cycle multiindex* list={
		mark list*
		\nextlist
		cyan,brown,green,blue,red
	}
	]
	
	\addplot table [x index={0}, y index={1}, col sep=comma]{data/qddwr/athletics.csv};
	\addplot table [x index={0}, y index={1}, col sep=comma]{data/qddwr/cycling.csv};
	
	\legend{athletics,cycling}
	
	\end{axis}
	\end{tikzpicture}
\end{figure}

\begin{figure}
	\caption{Shared queue and distributed database with replication - database throughput}
	\label{figure:queueddwr_database}
	\centering
	\begin{tikzpicture}
	\begin{axis}[
	title={Throughput against input rate a},
	width=0.9\textwidth,
	ybar=0pt,
	bar width=.02\textwidth,
	xlabel={Rate a},
	ylabel={Throughput},
	xtick=data,
	ymin=0, ymax=6,
	legend pos=north west,
	ymajorgrids=true,
	grid style=dashed,
	]
	
	\addplot [pattern=north east lines, pattern color=blue] table [x index={0}, y index={1}, col sep=comma]{data/qddwr/dbsrv_1.csv};
	\addplot [fill=green] table [x index={0}, y index={1}, col sep=comma]{data/qddwr/dbsrv_2.csv};
	\addplot [pattern=crosshatch, pattern color=brown] table [x index={0}, y index={1}, col sep=comma]{data/qddwr/dbsrv_3.csv};
	
	\legend{dbsrv1,dbsrv2,dbsrv3}
	
	\end{axis}
	\end{tikzpicture}
\end{figure}

%
% ---- System comparison
%
\FloatBarrier
\subsection{Comparison}

As the experiments on the three system models all used the same range of input rates {\itshape a} and {\itshape c}, and the throughput of {\itshape athletics} and {\itshape cycling} requests was recorded for each system, then their experimental results may be compared with each other. (Caveat: only the distributed database with replication model has the additional {\itshape diving} activity, but its throughput was identical to diving).  The {\itshape db} rate of the simple microservices model as been set back to the same as the other models i.e. 5.0 and the numerical results of each system model test are compared in Table \ref{table:comparison_results}.  Figure \ref{figure:comparison_chart} shows how the athletics and cycling throughputs behave for each system in graphical form.  If the models prove to be good representations of built systems, then such comparison shows the strengths and weaknesses of particular distributed system architectures and provides information for decision making in their selection during design.

\begin{table}[h!]
	\begin{center}
		\caption{Comparison of system results}
		\label{table:comparison_results}
		\pgfplotstabletypeset[
		col sep=comma,
		string type,
		columns/a/.style={column name=a, column type={p{.1\textwidth}}},
		columns/amicro/.style={column name=athletics, column type={p{.15\textwidth}}},
		columns/cmicro/.style={column name=cycling, column type={p{.15\textwidth}}},
		columns/aqddnr/.style={column name=athletics, column type={p{.15\textwidth}}},
		columns/cqddnr/.style={column name=cycling, column type={p{.15\textwidth}}},
		columns/aqddwr/.style={column name=athletics, column type={p{.15\textwidth}}},
		columns/cqddwr/.style={column name=cycling, column type={p{.15\textwidth}}},
		every head row/.style={before row=\hline Rate & \multicolumn{2}{l}{Simple Microservices} & \multicolumn{2}{l}{Queue + Distributed DB} & \multicolumn{2}{l}{Queue + DB with Replication} \\,after row=\hline},
		every last row/.style={after row=\hline},
		]{data/compare.csv}
	\end{center}
\end{table}

\paragraph{Athletics Throughput.}  The comparison shows that the actual throughput of the skewed demand for athletics tickets is lowest for the simple microservices system.  Introducing queue middleware to the system adds decoupling, resulting in a throughput much closer to the maximum service rate of a database node.  Using a distributed database with replication has a load sharing effect so that this system comes closest of all to a throughput that meets the demand.

\paragraph{Cycling Throughput.}  In contrast the cycling throughput is overall higher for the simple microservices system.  At lower levels of athletics demand, the queue-based systems win out, but as the skewed demand increases then it begins to have an impact on the cycling throughput in these systems.  As the microservices architecture isolates cycling from athletics, it remains at a constant rate.

\begin{figure}
	\caption{Comparison of systems - athletics and cycling throughput}
	\label{figure:comparison_chart}
	\centering
	\begin{tikzpicture}
	\begin{axis}[
	title={Throughput of athletics against input rate a},
	width=0.9\textwidth,
	xlabel={Rate a},
	ylabel={Throughput},
	xmin=0, xmax=10,
	ymin=0, ymax=10,
	legend pos=north west,
	legend cell align={left},
	ymajorgrids=true,
	grid style=dashed,
	cycle multiindex* list={
		mark list*
		\nextlist
		cyan,brown,green,blue,red
	}
	]
	
	\addplot table [x index={0}, y index={1}, col sep=comma]{data/micro/athletics.csv};
	\addplot table [x index={0}, y index={1}, col sep=comma]{data/qddnr/athletics.csv};
	\addplot table [x index={0}, y index={1}, col sep=comma]{data/qddwr/athletics.csv};
	
	\legend{Simple Microservices,Queue + Distributed DB,Queue + Distributed DB with Replication}
	
	\end{axis}
	\end{tikzpicture}
	
	\begin{tikzpicture}
	\begin{axis}[
	title={Throughput of cycling against input rate a},
	width=0.9\textwidth,
	xlabel={Rate a},
	ylabel={Throughput},
	xmin=0, xmax=10,
	ymin=0, ymax=1.2,
	legend pos=south west,
	legend cell align={left},
	ymajorgrids=true,
	grid style=dashed,
	cycle multiindex* list={
		mark list*
		\nextlist
		cyan,brown,green,blue,red
	}
	]
	
	\addplot table [x index={0}, y index={1}, col sep=comma]{data/micro/cycling.csv};
	\addplot table [x index={0}, y index={1}, col sep=comma]{data/qddnr/cycling.csv};
	\addplot table [x index={0}, y index={1}, col sep=comma]{data/qddwr/cycling.csv};
	
	\legend{Simple Microservices,Queue + Distributed DB,Queue + Distributed DB with Replication}
	
	\end{axis}
	\end{tikzpicture}
\end{figure}

%
% ---- Built systems sub-document
%

%
% ---- Built systems
%

\section{Built systems}

Reference to github at \cite{RN1073}

General design decisions:

Cassandra \cite{RN1050}\cite{RN1075} database.

Measurement using Coda Hale Metrics \cite{RN1079}.

Load testing using Apache JMeter \cite{RN1074}.

%
% ---- Simple microservices
%
\subsection{Simple microservices}

RESTful APIs using Java Spring \cite{RN1076}.

%
% ---- Shared queue middleware
%
\subsection{Shared queue middleware}

%
% ---- Distributed database with replication
%
\subsection{Distributed database with replication}



%
% ---- Conclusion sub-document
%

%
% ---- Technologies
%

\section{Conclusion and Future Work}

...

An interesting area of future work might be in using the modelling techniques in adaptive algorithms.  A model might be used as a policy for elastic scaling, and compared with the performance of other right-sizing strategies; control theory, machine learning and other model based techniques including statistical.

%
% ---- Bibliography ----
%

\bibliographystyle{splncs03}
\bibliography{references}

\end{document}