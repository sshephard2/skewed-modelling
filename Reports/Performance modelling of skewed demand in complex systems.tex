\documentclass{llncs}
\usepackage{graphicx,amssymb,pepa}

\begin{document}
\title{Performance modelling of skewed demand in complex systems}

\author{Stephen Shephard}

\institute{School of Computing Science, Newcastle University, Newcastle upon Tyne, NE1 7RU\\
	\email{s.shephard2@newcastle.ac.uk}}

\maketitle

%
% ---- Abstract and keywords
%

\begin{abstract}
On-line Transaction Processing (OLTP) applications must frequently deal with the issue of skewed demand for some resources.  This demand may overwhelm the whole system, affecting the owner's reputation and revenue.  In this article we present a ticketing use case and argue that at each layer of the architecture, the distributed computing technologies of the Cloud may maintain throughput to the lower demand resources, maximising the available functionality of the system.  
	\keywords{Cloud, middleware, microservices, distributed databases, load-balancing, performance}
\end{abstract}

%
% ---- Introduction
%

\section{Introduction}

There are many high-profile examples of whole IT systems brought down by customer demand for part of their services.  Customers were prevented from using any part of the London 2012 Olympic ticketing website on launch day to avoid demand overloading the system \cite{telegraph2011olympics}.  HBO Go was brought down by demand for the finale of ``True Detective'' \cite{hbo2014}.  Apple's iTunes Store suffered outage on the launch day of the iPhone 7 (new iPhone registration is carried out via an iTunes function) \cite{itunes2016}.

I propose that it is possible to design and build more resilient systems through effective use of Cloud technologies where higher than normal demand for one function or type of resource would not block access to the others. The London 2012 Olympics have now passed into history, but such a ticketing system would have many applications today, and may be generalised to a system for allocating and releasing other resources with variable demand.

%
% ---- Background
%

\section{Background}

I will consider an example system based on the Olympic ticketing use case above.  Tickets will be for a multi-sport event, and each will consist of a ticket type (the sport), date, row, and seat number.  The system will support three operations:
\begin{enumerate}
\item Search (for available tickets)
\item Book (allocate a ticket to a customer)
\item Return (customer releases a ticket allocation)
\end{enumerate}

If it is not possible to search for or book tickets of one type because some component is overloaded due to demand, then the system should allow booking of other ticket types.  It must also always be possible to return tickets of any type.

In this article, I am considering the problem of high demand for types of ticket.  This is not about the number of tickets available and I will not consider issues of fair allocation of scarce resources.  The demand may be:
\begin{enumerate}
\item predictable - we know which functions or ticket types are going to have the highest demand; or
\item unknown - we discover the areas of highest demand once the application goes online.
\end{enumerate}

\paragraph{System Architecture.}

The proposed system will use a distributed architecture.  Users will access it from a web-based front end.  Tickets will be stored in a database partitioned across several data nodes.  In between the web servers and database will be a number of worker applications to service user requests, connected to the web servers by some middleware.

\begin{figure}
\caption{Ticketing application distributed architecture}
\centering
\includegraphics[trim = 5 5 5 5, clip, width=\textwidth]{img/application}
\end{figure}

%
% ---- Technology Review sub-document
%

\include{techreview}

%
% ---- Modelling sub-document
%

\section{Modelling}

We can decouple worker applications from the front-end using asynchronous middleware.  Shared middleware balances the load, microservice architecture isolates it.  The system can adapt to current demand by using elastic scaling to create or destroy worker applications, and by using scaling groups we can ensure that the number of each application type is appropriate to the demand.

With care, we can use horizontal database partitioning to ensure that functions and/or data types are not shared between data nodes, isolating their demand from each other.

At the component level we can see whether an approach will balance or isolate load, or adapt to it, but at the system level we will need modelling techniques to predict the end to end throughput.

\paragraph{CloudSim.}  CloudSim \cite{RN69} is a Java framework for developing cloud datacentre simulations.  Much of it is concerned with modelling the efficient running of that infrastructure, for example the power usage, but it also includes utilisation models and may be useful for predicting the effect of elastic scaling.

CloudSim simulations require Java development for creation and modification, which is an overhead in building the models but offers more flexibility in applying them.  Process Algebra has closed-form solutions, though there is a PEPA Workbench tool \cite{RN51} that allows PEPA specifications to be parsed and run like programs, aiding experimentation on a range of action rates by automating repetitive calculations.  Both currently have their place as they predict different quantities of interest.

\paragraph{Process Algebra.} Process Algebras (such as PEPA or TIPP \cite{RN64}) allow us to model throughput in interdependent processes, with a mixture of independent and shared actions operating at different rates.  Each of our components can be described in this way, and queues have already been extensively modelled in PEPA \cite{RN75}.  The nature of process algebra as a mathematical language also means that it is possible to build a model of a whole system by composition of the component models.

\begin{figure}
\caption{PEPA queue model}
\centering
% Automatically generated by PEPA2Latex
% --begin
\begin{displaymath}
	\begin{array}{rcl}
%[0.0ex]		
\mathit{Website} & \rmdef & (\mathit{request},\mathit{r}).\mathit{Website}\\
		\mathit{Worker} & \rmdef & (\mathit{service},\mathit{s}).\mathit{Worker}\\
		\mathit{Queue_{0}} & \rmdef & (\mathit{request},\mathit{r}).\mathit{Queue_{1}}\\
		\mathit{Queue_{1}} & \rmdef & (\mathit{service},\mathit{s}).\mathit{Queue_{0}}\\
[0.0ex]		\multicolumn{3}{l}{\mathit{Website}\sync{request}\mathit{Queue_{0}}[N]\sync{service}\mathit{Worker}}\\
[0.0ex]	\end{array}
\end{displaymath}
% --end
\end{figure}


%
% ---- Conclusion sub-document
%

%
% ---- Technologies
%

\section{Conclusion and Future Work}

...

An interesting area of future work might be in using the modelling techniques in adaptive algorithms.  A model might be used as a policy for elastic scaling, and compared with the performance of other right-sizing strategies; control theory, machine learning and other model based techniques including statistical.

%
% ---- Bibliography ----
%

\bibliographystyle{splncs03}
\bibliography{project}

\end{document}